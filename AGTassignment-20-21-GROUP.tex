\documentclass{article}
\usepackage{braket, multirow, graphicx}

\addtolength{\oddsidemargin}{-.875in}
	\addtolength{\evensidemargin}{-.875in}
	\addtolength{\textwidth}{1.75in}

	\addtolength{\topmargin}{-.875in}
	\addtolength{\textheight}{1.75in}

\begin{document}

\begin{center}
	\LARGE{Algorithmic Game Theory Summative Assignment -- Group work Component}\\[0.1cm]
	\Large{Dr Eleni Akrida}\\[0.1cm]
	2020\\[0.5cm]
\end{center}


\noindent Each group should submit 2 files:
	\begin{enumerate}
		\item a PDF file named ReportGroupXX, and
		\item an MP3 file named PodcastGroupXX, 
	\end{enumerate} 
where XX is your group number, e.g. 03 or 15. \textbf{Only one member of the group should submit the above files.}\\

\noindent \textbf{Your report should be written in LaTeX;} you may use the settings of the LaTeX source file provided alongside this pdf.\\
\noindent One way to create your \textbf{podcast} would be for all members to join a Zoom meeting where you will record the podcast and afterwards extract the audio of the recording.\\[0.5cm]



\textbf{\underline{GUIDANCE FOR GROUP MEETINGS:}}  

\begin{itemize}
\item Each group should aim to meet for a total of 10 hours, i.e. 2 hours per week of term or (roughly) 1 hour per week throughout the duration of the coursework. 
\item Aim to make the meetings productive, i.e, work should get done \emph{during} the meetings and these should not be just occasions on which you have a chat about what needs doing. Examples of things that groups can do during meetings include: discuss the references, draft the report, write the podcast etc.  
\item Before or during your first meeting, you should set aims for your upcoming meetings. 
\item Before each meeting, you should set an agenda and during each meeting you should set an action log for what needs to be done be each member before the next meeting.
\item Make sure that you agree in advance, e.g. during your first meeting, how you will be sharing documents so that you can work together in real-time. For LaTeX files, Overleaf (http://overleaf.com/) is a good option to share documents to be worked on by more than one person at the same time.
\end{itemize}
~\\





\noindent \textbf{There are two options for the combination of report and podcast that each group can create. Each group should select only one of the combinations to submit.}\\[1cm]



\noindent\underline{\textbf{OPTION 1:}}

\begin{itemize}
	\item \textbf{Report on the complexity of Nash equilibria:}\\Provide a combined synopsis of \cite{DGP} and \cite{FPT}. \hfill{\bf [25 marks]}\smallskip
	
	You should read the papers and present a condensed combination of them in up to 4 pages (including references) so as to provide:
	
	\begin{itemize}
		
		\item relevant background material;
		
		\item an overview of the results;
		
		\item intuition and/or description of reductions used to show the above results;
		
		\item any further discussion on related research.
		
	\end{itemize}

	\item \textbf{and Podcast on auctions:}\\Create a podcast that presents the subject of auctions in the context of game theory. \hfill{\bf [25 marks]}\smallskip
	
	Assume that your audience is Year 1 Computer Science students who are familiar with the definition of a Nash equilibrium. Your podcast should last a maximum of 12 minutes and should introduce the concept of auctions. You are expected to discuss theory and examples in order to teach your audience about auctions.
	
	Each member of the group is expected to speak in the audio, but there is no requirement to speak for at least a specific duration; however, if one member does not contribute much to the presentation of the podcast (speech), they are expected to contribute more towards other aspects of its creation, e.g. organisation of podcast, collection of material to be discussed etc.
	
	You will be assessed based on:
	
	\begin{itemize}
		\item your knowledge and understanding in the relevant subject matter;
		\item your argument and reasoning: 	clarity of thought, argument, analysis and use of evidence, integration of theory and/or practice, reflection;
		\item your oral presentation/commentary: design of podcast including language and style, structure, tailoring to context, clarity, organisation, referencing;
		\item your skills in the presentation: audibility, pace, timing, engagement.
	\end{itemize}
\end{itemize}





\noindent\underline{\textbf{OPTION 2:}}

\begin{itemize}
	\item \textbf{Report on the complexity of congestion games:}\\Provide a combined synopsis of \cite{FPT} and \cite{FKS}. \hfill{\bf [25 marks]}\smallskip
	
	You should read the papers and present a condensed combination of them in up to 4 pages (including references) so as to provide:
	
	\begin{itemize}
		
		\item relevant background material;
		
		\item an overview of the results;
		
		\item intuition and/or description of reductions and/or proofs used to show the above results;
		
		\item any further discussion on related research.
		
	\end{itemize}
	
	\item \textbf{and Podcast on Nash equilibria:}\\Create a podcast that presents the subject of (mixed and pure) Nash equilibria. \hfill{\bf [25 marks]}\smallskip
	
	Assume that your audience is Year 1 Computer Science students. Your podcast should last a maximum of 12 minutes and should introduce the concept of a Nash equilibrium. You are expected to discuss theory and examples in order to teach your audience about Nash equilibria.
	
	Each member of the group is expected to speak in the audio, but there is no requirement to speak for at least a specific duration; however, if one member does not contribute much to the presentation of the podcast (speech), they are expected to contribute more towards other aspects of its creation, e.g. organisation of podcast, collection of material to be discussed etc.
	
	You will be assessed based on:
	
	\begin{itemize}
		\item your knowledge and understanding in the relevant subject matter;
		\item your argument and reasoning: 	clarity of thought, argument, analysis and use of evidence, integration of theory and/or practice, reflection;
		\item your oral presentation/commentary: design of podcast including language and style, structure, tailoring to context, clarity, organisation, referencing;
		\item your skills in the presentation: audibility, pace, timing, engagement.
	\end{itemize}
\end{itemize}



\vfill

\begin{thebibliography}{99}

\bibitem{DGP} C. Daskalakis, P. W. Goldberg and C. H. Papadimitriou, The complexity of computing a Nash equilibrium, \emph{Commununications of the ACM} 52 (2009) 89--97.


\bibitem{FPT} A. Fabrikant, C. H. Papadimitriou, K. Talwar, The complexity of pure Nash equilibria, \emph{STOC} (2004) 604--612.

\bibitem{FKS} D. Fotakis, S. C. Kontogiannis, P. G. Spirakis, Selfish unsplittable flows, \emph{Theoretical Computer Science} 348 (2005) 226--239.

\end{thebibliography}


\end{document}

