
\documentclass{article}
\usepackage{multirow, graphicx, amsfonts, amsmath}

\addtolength{\oddsidemargin}{-.875in}
	\addtolength{\evensidemargin}{-.375in}
	\addtolength{\textwidth}{1.55in}

	\addtolength{\topmargin}{-.375in}
	\addtolength{\textheight}{1.75in}

\begin{document}

\begin{center}
	\LARGE{Algorithmic Game Theory Summative Assignment -- Individual Component}\\[0.1cm]
	\Large{lrfk99}\\[0.1cm]
	2021\\[0.5cm]
\end{center}

\begin{enumerate}
	
	\item[\textbf{Exercise 1.}]   %%%%%%%  11 MARKS  %%%%%%% 
	
	Consider the following instance of the load balancing game where the number of tasks is equal to the number of machines, and in particular we have:
	\begin{itemize}
		\item $m$ identical machines $M_1, M_2, \dots, M_m$ (all of speed 1),
		\item $m$ identical tasks $w_1 = w_2 = \dots = w_m = 1$.
	\end{itemize}
	Consider also the mixed strategy profile $A$ where each of the tasks is assigned to all machines equiprobably (i.e. with probability $1/m$). 
	\begin{enumerate}
		\item[(a)] Calculate the ratio $cost(A)/cost(OPT)$ in the special case where $m=2$.  \hfill{\bf [3 marks]}\smallskip
		\item[(b)] Calculate the ratio $cost(A)/cost(OPT)$ in the special case where $m=3$.  \hfill{\bf [3 marks]}\smallskip
		\item[(c)] Discuss what this ratio is for arbitrary $m$. What does this imply about the Price of Anarchy on identical machines for mixed Nash equilibria?  \hfill{\bf [5 marks]}\smallskip
	\end{enumerate}
	\vspace*{0.8cm}
	

	\item[\textbf{Exercise 2.}]  %%%%%%%  10 MARKS  %%%%%%% 

	We consider a second-price sealed-bid auction where there are $n$ bidders who bid as follows:
	\begin{itemize}
		\item Bidders 1 up to $n-1$ bid either 1 dollar or $r > 1$ dollars equiprobably and
		independently of the rest.
		\item Bidder $n$ bids $h$ dollars, where $h > r$.
	\end{itemize}
	The seller's expected revenue $R$ is the expectation of the second highest value. 
	\begin{itemize}
		\item[(a)] What is the value that $R$ is approaching when $n$ is very large? \hfill{\bf [1 marks]}\smallskip
		\item[(b)] Justify your answer by taking the limit. \hfill{\bf [9 marks]}\smallskip
	\end{itemize}

	\newpage




	\item[\textbf{Exercise 3.}]   %%%%%%%  15 MARKS  %%%%%%% 

Mary and Alice are buying items for Sunday lunch. Mary buys either chicken $(C)$ or beef $(B)$ for the main course and Alice buys either juice $(J)$ 
or wine $(W)$. Both people prefer wine with beef and juice with chicken. The opposite alternatives are equally displeasing.
However, Mary prefers beef over chicken, while Alice prefers chicken over beef.

We assume that Mary buys first and then tells Alice what she bought,
so when Alice makes her decision, she knows if the main course is beef or chicken.

\begin{enumerate}
\item[(a)] Express the above preferences as payoffs by using numbers\\(e.g.
 $u_M(B,W) = 2$, $u_A(B,W) = \ldots$  etc.)               \hfill{\bf [2 marks]}\smallskip
\item[(b)] Write down a bimatrix game with Mary as the row player 
and Alice as the column player, using your chosen payoffs.           \hfill{\bf [4 marks]}\smallskip
\item[(c)] Write down a game tree representing this game as an extended game.  \hfill{\bf [4 marks]}\smallskip
\item[(d)] Find a solution for the extended game using backward induction.\\Describe your steps.  \hfill{\bf [5 marks]}\smallskip
\end{enumerate}
	\vspace*{0.8cm}


	
	\item[\textbf{Exercise 4.}]  %%%%%%%  12 MARKS  %%%%%%% 
	
	\begin{enumerate}
		\item[(a)] Price of $1^{st}$ item: $k-1$, price of every other item: $0$. Buyer $x_1$ gets the $1^{st}$ item at price $k-1$. 
		\item[(b)] Using the bipartite graph auction procedure. 
		In the first round where all prices are zero, 
		after drawing the preferred-seller graph according to the utilities of the buyers: $u_{ij} = v_{ij} - p_i$, 
		there is a constricted set in the buyers' set (all buyers). 
		The neighbourhood of the constricted set is item 1, intuitively every buyer prefers item 1. 
		Therefore the price of item 1 is increased by 1, to 1. 
		Every other price remains 0 since it is not in the neighbourhood of a constricted set, 
		so the minimum price remains 0 and there is no need to update every price by subtracting the minimum from each. 
		The previous steps are repeated $k-1$ times, and the size of the constricted set in the buyers' set decreases by 1 every round 
		(round $i$ removes buyer $x_{k-i+1}$ from the constricted set). 
		In every round, the neighbourhood of the constricted set is item 1, so after every round the price of item 1 increases by 1. 
		Therefore, after round $i$, the price of item 1 is $i$, and after round $k-1$, the price of item 1 is $k-1$. 
		In round $k$, after drawing the preferred-seller graph according to the utilities of the buyers: $u_{ij} = v_{ij} - p_i$, 
		there is a perfect matching where every buyer $x_i$ buys item $i$. 
		The price of item 1 at this point is $k-1$, and the price of every other item is $0$. 
		Buyer $x_1$ gets the $1^{st}$ item in the perfect matching since their payoff for this is $1$ ($k-(k-1)$), 
		and their payoff for every other item is $0$ ($0-0$). 
		So buyer $x_1$ get the $1^{st}$ item at price $k-1$. 
		\item[(c)] Second-price sealed-bid auction of a single item. 
		The buyer with the highest valuation gets the item (item 1), and pays the second highest-valuation. 
		All other items are fake additional items created in order to model the single-item auction as a bipartite graph auction. 
	\end{enumerate}


\end{enumerate}
\end{document}
