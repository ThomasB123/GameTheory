
\documentclass{article}
\usepackage{multirow, graphicx, amsfonts, amsmath, istgame}

\addtolength{\oddsidemargin}{-.875in}
\addtolength{\evensidemargin}{-.375in}
\addtolength{\textwidth}{1.55in}

\addtolength{\topmargin}{-.375in}
\addtolength{\textheight}{1.75in}

\begin{document}

\begin{center}
	\LARGE{Algorithmic Game Theory Summative Assignment \\Individual Component}\\[0.1cm]
	\Large{LRFK99}\\[0.1cm]
	2021\\[0.5cm]
\end{center}

\begin{enumerate}
	
	\item[\textbf{Exercise 1.}]   %%%%%%%  11 MARKS  %%%%%%%
	
	\begin{enumerate}
		\item[(a)]
		\begin{equation*}
			cost(A) = \mathbb{E}[\max_{j \in [m]}{(\ell_j)}] = \frac{1}{4}(1 \cdot 2 + 2 \cdot 2) = 1.5
		\end{equation*}
		\begin{equation*}
			cost(OPT) = 1
		\end{equation*}
		\begin{equation*}
			cost(A)/cost(OPT) = 1.5/1 = 1.5
		\end{equation*}
		\item[(b)]
		\begin{equation*}
			cost(A) = \mathbb{E}[\max_{j \in [m]}{(\ell_j)}] = \frac{1}{27}(1 \cdot 6 + 2 \cdot 18 + 3 \cdot 3) = 1.\dot{8}
		\end{equation*}
		\begin{equation*}
			cost(OPT) = 1
		\end{equation*}
		\begin{equation*}
			cost(A)/cost(OPT) = 1.\dot{8}/1 = 1.\dot{8}
		\end{equation*}
		\item[(c)]
		For arbitrary $m$, this ratio is the Price of Anarchy, 
		and it is equal to the social cost of the mixed strategy profile $A$ ($cost(A)$) since the optimal social cost ($cost(OPT)$) is always $1$ for any $m$. 
		Therefore the ratio depends only on the social cost of the mixed strategy, which increases logarithmically as $m$ increases. 

		This implies that the Price of Anarchy on identical machines for mixed Nash equilibria is $O(\frac{\log m}{\log \log m})$, 
		it increases logarithmically as the number of machines and tasks increase. 
	\end{enumerate}
	

	\item[\textbf{Exercise 2.}]  %%%%%%%  10 MARKS  %%%%%%% 

	\begin{itemize}
		\item[(a)] $r$
		\item[(b)]
		\begin{equation*}
			\begin{split}
				R 
				& = 1 \cdot (\frac{1}{2})^n + r \cdot (1-(\frac{1}{2})^n) \\
				therefore \quad \lim_{n \to \infty} R
				& = \lim_{n \to \infty} [1 \cdot (\frac{1}{2})^n + r \cdot (1-(\frac{1}{2})^n)] \\
				& = 1 \cdot (\lim_{n \to \infty}(\frac{1}{2})^n) + r \cdot (1-(\lim_{n \to \infty}(\frac{1}{2})^n)) \\
				& = 1 \cdot 0 + r \cdot (1-0), \quad since \quad \lim_{n \to \infty}(\frac{1}{2})^n = \lim_{x \to \infty}(\frac{1}{x}) = 0 \\
				& = 0 + r \cdot 1 \\
				& = 0 + r \\
				& = r 
			\end{split}
		\end{equation*}
	\end{itemize}


	\item[\textbf{Exercise 3.}]   %%%%%%%  15 MARKS  %%%%%%%
	
	\begin{enumerate}
		\item[(a)]
		$u_M(C,J) = 1$, $u_A(C,J) = 2$, $u_M(C,W) = -2$, $u_A(C,W) = -1$, \\
		$u_M(B,J) = -1$, $u_A(B,J) = -2$, $u_M(B,W) = 2$, $u_A(B,W) = 1$
		\item[(b)]
		$A = \begin{bmatrix}
			1 & -2 \\
			-1 & 2
		\end{bmatrix}$ \quad
		$B = \begin{bmatrix}
			2 & -1 \\
			-2 & 1
		\end{bmatrix}$
		\item[(c)]\textit{}
		\begin{center}
			\begin{istgame}
				\xtdistance{20mm}{60mm}
				\istroot(0)[chance node]{Mary}
				\istb{Chicken}[al]
				\istb{Beef}[ar]
				\endist
				\xtdistance{20mm}{30mm}
				\istroot(A)(0-1)<150>{Alice}
				\istb{Juice}[al]{1,2}
				\istb{Wine}[ar]{-2,-1}
				\endist
				\istroot(B)(0-2)<30>{Alice}
				\istb{Juice}[al]{-1,-2}
				\istb{Wine}[ar]{2,1}
				\endist
			\end{istgame}
		\end{center}
		\item[(d)]
		Mary will buy beef and Alice will subsequently buy wine. \\
		Mary can reason that if she buys beef then Alice will buy wine, because doing so is better for her than buying juice. \\
		Given that Alice will respond to beef in this way, Mary is better off buying beef. \\
		If Mary were to buy chicken then she can reason that Alice would buy juice because doing so is better for her than buying wine, 
		this results in a payoff of $1$ for Mary, which is less than the payoff of $2$ she will get if she buys beef.
	\end{enumerate}

	
	\item[\textbf{Exercise 4.}]  %%%%%%%  12 MARKS  %%%%%%% 
	
	\begin{enumerate}
		\item[(a)] Price of $1^{st}$ item: $k-1$, price of every other item: $0$. Buyer $x_1$ gets the $1^{st}$ item at price $k-1$. 
		\item[(b)] Using the bipartite graph auction procedure. 
		In the first round where all prices are zero, 
		after drawing the preferred-seller graph according to the utilities of the buyers: $u_{ij} = v_{ij} - p_i$, 
		there is a constricted set in the buyers' set (all buyers). 
		The neighbourhood of the constricted set is item 1, intuitively every buyer prefers item 1. 
		Therefore the price of item 1 is increased by 1, to 1. 
		Every other price remains 0 since it is not in the neighbourhood of a constricted set, 
		so the minimum price remains 0 and there is no need to update every price by subtracting the minimum from each. \\
		The previous steps are repeated $k-1$ times, and the size of the constricted set in the buyers' set decreases by 1 every round 
		(round $i$ removes buyer $x_{k-i+1}$ from the constricted set). 
		In every round, the neighbourhood of the constricted set is item 1, so after every round the price of item 1 increases by 1. 
		Therefore, after round $i$, the price of item 1 is $i$, and after round $k-1$, the price of item 1 is $k-1$. 
		In round $k$, after drawing the preferred-seller graph according to the utilities of the buyers: $u_{ij} = v_{ij} - p_i$, 
		there is a perfect matching where every buyer $x_i$ buys item $i$. 
		The price of item 1 at this point is $k-1$, and the price of every other item is $0$. \\
		Buyer $x_1$ gets the $1^{st}$ item in the perfect matching since their payoff for this is $k-(k-1)=1$, 
		and their payoff for every other item is $0-0=0$. 
		So buyer $x_1$ get the $1^{st}$ item at price $k-1$. 
		\item[(c)] Second-price sealed-bid auction of a single item. 
		Where all other items are fake additional items created in order to model the single-item auction as a bipartite graph auction. 
		The dominant strategy of all buyers is to bid truthfully, 
		so the buyer with the highest valuation gets the item (item 1), and pays the second highest-valuation. 
	\end{enumerate}


\end{enumerate}
\end{document}
