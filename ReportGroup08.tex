
\documentclass{article}
\usepackage{braket, multirow, graphicx}

\addtolength{\oddsidemargin}{-.875in}
	\addtolength{\evensidemargin}{-.875in}
	\addtolength{\textwidth}{1.75in}

	\addtolength{\topmargin}{-.875in}
	\addtolength{\textheight}{1.75in}

\begin{document}

\begin{center}
	\LARGE{Algorithmic Game Theory Summative Assignment -- Group work Component}\\[0.1cm]
	\Large{Group 8}\\[0.1cm]
	2021\\[0.5cm]
\end{center}

\begin{itemize}
	\item \textbf{Report on the complexity of Nash equilibria:}\\Provide a combined synopsis of \cite{DGP} and \cite{FPT}. \hfill{\bf [25 marks]}\smallskip
	
	You should read the papers and present a condensed combination of them in up to 4 pages (including references) so as to provide:
	
	\begin{itemize}
		
		\item relevant background material;
		
		\item an overview of the results;
		
		\item intuition and/or description of reductions used to show the above results;
		
		\item any further discussion on related research.
		
	\end{itemize}

	\item \textbf{and Podcast on auctions:}\\Create a podcast that presents the subject of auctions in the context of game theory. \hfill{\bf [25 marks]}\smallskip
	
	Assume that your audience is Year 1 Computer Science students who are familiar with the definition of a Nash equilibrium. Your podcast should last a maximum of 12 minutes and should introduce the concept of auctions. You are expected to discuss theory and examples in order to teach your audience about auctions.
	
	Each member of the group is expected to speak in the audio, but there is no requirement to speak for at least a specific duration; however, if one member does not contribute much to the presentation of the podcast (speech), they are expected to contribute more towards other aspects of its creation, e.g. organisation of podcast, collection of material to be discussed etc.
	
	You will be assessed based on:
	
	\begin{itemize}
		\item your knowledge and understanding in the relevant subject matter;
		\item your argument and reasoning: 	clarity of thought, argument, analysis and use of evidence, integration of theory and/or practice, reflection;
		\item your oral presentation/commentary: design of podcast including language and style, structure, tailoring to context, clarity, organisation, referencing;
		\item your skills in the presentation: audibility, pace, timing, engagement.
	\end{itemize}
\end{itemize}



\vfill

\begin{thebibliography}{99}

\bibitem{DGP} C. Daskalakis, P. W. Goldberg and C. H. Papadimitriou, The complexity of computing a Nash equilibrium, \emph{Commununications of the ACM} 52 (2009) 89--97.


\bibitem{FPT} A. Fabrikant, C. H. Papadimitriou, K. Talwar, The complexity of pure Nash equilibria, \emph{STOC} (2004) 604--612.

\bibitem{FKS} D. Fotakis, S. C. Kontogiannis, P. G. Spirakis, Selfish unsplittable flows, \emph{Theoretical Computer Science} 348 (2005) 226--239.

\end{thebibliography}


\end{document}
